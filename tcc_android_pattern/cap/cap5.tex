\chapter{Conclusão}

\section{Discussão dos Resultados}
As classes destinadas à implementação da interface no framework android fornecem
acceso a recursos que servem para implementação de responsabilidades não
relacionadas com a interface. Essa característica do framework android leva à
implementação da camada de View com diversas responsabilidades que não são
inerentes à interação com o usuário. Isso dificultou a refatoração, pois as
classes que exercem o papel de Presenter necessitam interagir com as classes de
View para acessar esses recursos, além de atualizar o estado da View. O uso do
padrão de injeção de dependência\footnote{\url{http://en.wikipedia.org/wiki/Dependency_injection}}
pode ser aplicado para acessar esses recursos e serviços sem a necessidade de
interação com a classe de Veiw.

% danny - o parag. comeca estranho... parece q vc vai citar o padrao, mas ae
% começa a falar sobre granularidade... tenta conectar melhor as ideias. releia
% o parag. e tente se convencer q esta facil de entender.
Ao seguir o padrão conforme descrito por \citeonline{fowler:ui}, a granularidade dos
métodos aumenta pois cada um dos componentes implementarão uma parte do caso de
uso aumentando o números de métodos que se reflete na métrica WMC, isso afeta
também a métrica RFC pois esses métodos estão relacionados entre si, aumentando
a troca de mensagens entre a view e o presenter. Um ponto positivo sobre isso é o
fato de que métodos  complexos com muitas linhas de código foram desmembrados em
métodos menores e mais simples, implementados tanto na View como no Presenter.

Houve diminuição na métrica CBO nas classes alteradas pois diversas
responsabilidades que utilizam essas dependências foram movidas para a classe de
Presenter. Analisando de forma geral, essas dependências permanecem no pacote
além de ser criada mais um acoplamento entre a View e a nova classe Presenter.
Os experimentos demonstraram que a aplicação do padrão MVP promoveu de forma
significativa maior coesão no aplicativo.

A métrica WMC está relácionada com as métricas DIT e NOC. Como houve pouca
variação no DIT e nenhuma variação no NOC o aumento da WMC não tem impacto
relevante. 

Entretanto a métrica RFC aumentou indicando maior complexidade. Os resultaddos
mostram que isso se deve à melhora da coesão do código demonstrado pela
diminiução da métrica LCOM. Comforme um software é desenvolvido novas classes e
troca de menssagens são implementadas afetando as métricas WMC e RFC. Como não
houve uma refatoração mais extensiva do projeto não é possível determinar
valores de referência para as métricas. Não é possível relacionar diretamente a
métrica CBO com base na variação dos resultados para esta métrica, apesar de
que, os valores se mantiveram abaixo dos encontrados da versão de referência.
Isso requer mais estudos abordando outros padrões de projeto. Portanto é possível
determinar que a arquitetura proposta melhorou a qualidade do objeto de estudo
porque esta mais coeso.


\section{Trabalhos Futuros}

Existem outros padrões de projetos para o desenvolvimento da camada de
apresentação de um software que não foram analisados nesse trabalho, a saber: 
MVVM, MVP-VM, MVPC. Essas variações no padrão MVC surgiram em contextos
diversificados e podem agregar algum benefício à qualidade do aplicativo.
Este trabalho não aborda o impacto do padrão MVP em outras métricas de qualidade
de código.
Não foi feita uma avaliação dos impactos na performance do aplicativo devido ao
uso do padrão MVP. A inclusão de mais objetos interagindo trocando mensagens
pode depreciar a performance, levando-se em consideração sua execução em
ambientes mais restritos, como no caso de um aparelho móvel.