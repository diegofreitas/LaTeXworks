\chapter{Resultados e Conclusões}

\section{Conclusões}

É um equívoco MVC no contexto atual pois o padrão não se adequa. o que existe
são evoluções aplicadas a uma determinada situação desenvolvimento web, desktop
e 

abordar quais seriam os parâmetros aceitáveis para as métricas(difícil afirmar
quais são esses parametros, pois cada ferramenta de coleta tem sua propria
forma de calcular. citar artigo sobre isso)

Validade dessas métricas, talvez o ideal fosse customizar essas métricas para o
framework?;


Possível aumento do código com essa divisão com mais classes --- Isso ocorreu



O presenter deveria ser independente de tecnologia mas na plataforma android,
seguir isso a risca iria criar uma aberração pois a activity ou fragmentos
proveem muitos recursos usados para diversos interesses(View
Controller/Presenter Model), por isso é tão natural esses componentes serem
implementados com diversas responsabilidades.

O componente Loader é dificil de refatorar, pois esta muito acoplado com a view!

O foco da refatoração foi remover o model e o estado da view para o presenter

Quais metricas foram afetadas.(todas exceto NOC e DIT)?
- WMC,CBO,RPC,LCOM foram afetadas

a refatoração melhorou cbo/lcom e piorou ou manteve rpc wmc, 




\begin{description}
  \item[WMC] Segundo \citeonline{cksuite}  ``\ldots The
larger the number of methods in a class the greater the potential impact on children, since children will
inherit all the methods defined in the class. \ldots Classes with large numbers of methods are
likely to be more application specific, limiting the possibility of
reuse.'' as classes analisadas são de inteface, portanto implementam uma
interação com o usuário bem específica. não havera impacto negativo, além disso
a quantidade de métodos aumentou mas com uma complexidade menor pois métodos com
várias estruturas de controle forem quebrados em métodos menores, podendo ser
avaliado como positivo.

\end{description}

Houve aumento no valor da métrica DIT, que deve ser mantido baixo pois quanto
mais abaixo na hierarquia de heraça a classe estiver, menos previsível será seu
comportamento devido a quantidade de classes acima, entretanto o aumento no DIT
se deve ao fato que o ckjm considera a implementação de interface como herança.
A interface define somente o contrato que a classe de implementar não havendo
nenhuma implementação que pudesse interferir no comportamento da classe,
portanto essa alteração no DIT não deve ser considerada.




\section{Trabalhos Futuros}

Existem outros padrões de projetos para o desenvolvimento da camada de
apresentação de um software que não foram analisados nesse trabalho: MVVM,
MVP-VM, MVPC.

Desenvolver métricas


Este trabalho não fez uma avaliação dos impactos na performance do aplicativo
devido ao uso desses padrões. A inclusão de mais objetos interagindo,
redirecionando mensagens pode depreciar a performance.
