\chapter{Conclusão}

As classes destinadas à implementação da interface no framework android fornecem
acceso a recursos que servem para implementação de responsabilidades não
relacionadas com a interface.

Essa característica do framework android leva à implementação da camada de View
com diversas responsabilidades que não são inerentes à interação com o
usuário. Isso dificultou a refatoração pois as classes que exercem o papel de
presenter necessitam interagir com as classes de view para acessar esses
recursos, além de atualizar o estado da view. O uso do padrão de injeção de
dependência\footnote{http://en.wikipedia.org/wiki/Dependency_injection} pode ser
aplicado para acessar esses recursos e serviços sem a necessidade de interação com a classe de Veiw.

Ao seguir o padrão conforme descrito por \citeonline{fowler:ui}, a granularidade dos
métodos aumenta pois cada um dos componentes implementão uma parte do caso de
uso aumentando o números de métodos que se reflete na métrica WMC, isso afeta
também a métrica RFC pois esses métodos estão relacionados entre si aumentando a
troca de mensagens entre a view e o presenter. Um ponto positivo sobre isso é o
fato de que métodos  complexos com muitas linhas de código foram desmembrados em
métodos menores e mais simples implementados tanto na view como no presenter.

Houve diminuição na métrica CBO nas classes alteradas pois diversas
responsabilidades que utilizam essas dependências foram movidas para a classe de
presenter. Analisando de forma geral, essas dependências permanecem no pacote
além de ser criada mais uma acoplamento entre a view e a nova classe presenter.
Os experimentos demonstraram que a aplicação do padrão MVP promoveu de forma
significativa maior coesão no aplicativo objeto de estudo deste trabalho.


\section{Trabalhos Futuros}

Existem outros padrões de projetos para o desenvolvimento da camada de
apresentação de um software que não foram analisados nesse trabalho a saber: 
MVVM, MVP-VM, MVPC. Essas variações no padrão MVC surgiram em contextos
diversificados e podem agregar algum benefício à qualidade do aplicativo.
Este trabalho não aborda o impacto do padrão mvp em outras métricas de qualidade
de código.
Não foi feita uma avaliação dos impactos na performance do aplicativo devido ao
uso do padrão MVP. A inclusão de mais objetos interagindo trocando mensagens
pode depreciar a performance levando-se em consideração sua execução em
ambientes mais restritos como um aparelho móvel.
