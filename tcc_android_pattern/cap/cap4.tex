\chapter{Execução da Pesquisa}


\section{Ferramentas usadas}

O código será versionado no
Github\footnote{https://github.com/diegofreitas/platform_packages_apps_contacts}
onde será feito o gerenciamento das versões de cada iteração.
As ferramentas utilizadas para a refatoração serão a IDE Eclipse(Juno) com
plugin ADT v21 para facilitar a edição do código e ferramentas de construção
do projeto existentes no próprio repositório do android tendo em vista que todo
o processo de compilação e empacontamento não visa ser usado em uma ide.
Para fazer a coleta das métricas é necessario que a ferramenta analise código
java e contemple todas as métricas descritas na seção \ref{sec:metrics}. O
programa Chidamber and Kemerer Java Metrics\footnote{https://github.com/dspinellis/ckjm} atende esses
critérios, além de ser um projeto de código-aberto.

\section{Análise do objeto de estudo}

O aplicativo a ser refatorado tem funcionalidades para gerenciamento de
contatos. Dentro deste conjunto de casos de uso será feratorado o pacote
referente ao gerenciamento de grupos de contatos presente no pacote
\textbf{com.android.contacts.group} que contém componentes de tela para
interação com o usuário a saber:
\begin{description}
\item[GroupDetailFragment.java] Exibe os dados de um grupo de contatos.
\item[GroupBrowseListFragment.java] Fonece uma lista de grupos.
\item[GroupEditorFragment.java] Disponibiliza um formulário para edição dos
dados de um grupo.
\end{description}

Estas intefaces são usadas dentro de activities que controlam uma parte do fluxo
de interação e se comportam de forma diferente conforme o tipo de dispositivo
móvel utilizado. Devido a essa complexidade, não será feita nenhuma alteração na
interface pública dos componentes refatorados evitando efeitos colaterais em
outras partes do aplicativo.

Os componetes elencados contém código não somente relacionado com a lógica de
apresentação como também interagem diretamente com classes destinadas ao acesso
de dados e serviços existentes nas dependências do projeto, por exemplo,
gerenciamento de contas do usuário. A iteração de refatoração compreenderá na
na refatoração de cada um dos componentes descritos. O marco de referência de
dados das métricas presentes na tabela \ref{tab:baseline} será feita a partir da
versão \verb|4.4.2_r1| do aplicativo.
%be5bc44 

\begin{table}
	\centering
    \begin{tabular}{ | l | l | l | l |}
    \hline
    Métrica & Min. & Max. & Média \\ \hline
    WMC & 1 & 48  	& 8.5161290323   \\ \hline
    DIT	& 0	& 2		&0.7741935484\\ \hline
	NOC & 0	& 0		& 0\\ \hline
	CBO	& 0	& 50	& 10.1612903226\\ \hline
	RFC	& 1	& 157	& 23.7419354839\\ \hline
	LCOM	& 0		& 772	& 57.4838709677\\ \hline
    \end{tabular}
    \caption{Métricas CK para linha de base}
    \label{tab:baseline}
\end{table}



\section{Primeira Iteração}
%Commit 2501ad5

\begin{table}
	\centering
    \begin{tabular}{ | l | l | l | l |}
    \hline
    Métrica & Min. & Max. & Média \\ \hline
    WMC & 1 & 48  	& 8.5161290323   \\ \hline
    DIT	& 0	& 2		&0.7741935484\\ \hline
	NOC & 0	& 0		& 0\\ \hline
	CBO	& 0	& 50	& 10.1612903226\\ \hline
	RFC	& 1	& 157	& 23.7419354839\\ \hline
	LCOM	& 0		& 772	& 57.4838709677\\ \hline
    \end{tabular}
    \caption{Métricas CK para linha de base}
    \label{tab:baseline}
\end{table}

\section{Segunda Iteração:}

\begin{table}
	\centering
    \begin{tabular}{ | l | l | l | l |}
    \hline
    Métrica & Min. & Max. & Média \\ \hline
    WMC & 1 & 48  	& 8.5161290323   \\ \hline
    DIT	& 0	& 2		&0.7741935484\\ \hline
	NOC & 0	& 0		& 0\\ \hline
	CBO	& 0	& 50	& 10.1612903226\\ \hline
	RFC	& 1	& 157	& 23.7419354839\\ \hline
	LCOM	& 0		& 772	& 57.4838709677\\ \hline
    \end{tabular}
    \caption{Métricas CK para linha de base}
    \label{tab:baseline}
\end{table}

%Commit 26f473a

\section{Terceira Iteração:} 


\begin{table}
	\centering
    \begin{tabular}{ | l | l | l | l |}
    \hline
    Métrica & Min. & Max. & Média \\ \hline
    WMC & 1 & 48  	& 8.5161290323   \\ \hline
    DIT	& 0	& 2		&0.7741935484\\ \hline
	NOC & 0	& 0		& 0\\ \hline
	CBO	& 0	& 50	& 10.1612903226\\ \hline
	RFC	& 1	& 157	& 23.7419354839\\ \hline
	LCOM	& 0		& 772	& 57.4838709677\\ \hline
    \end{tabular}
    \caption{Métricas CK para linha de base}
    \label{tab:baseline}
\end{table}
%Commit ec314ac
