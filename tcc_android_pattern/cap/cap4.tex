\chapter{Conclusão} 

O presente trabalho avaliou os padrões de projetos Model View Controller e o
Model View Presenter mostrando em que contextos cada um se aplica para o
desenvolvimento da camada de apresentação de um sistema.
O padrão MVP define melhor as responsabilidades dos componentes para
implementação de interfaces levando em consideração os modelo de programação
usado no framework android onde as responsabilidades do Controller(definidos no
padrão MVC) estão implementadas nos próprios componentes visuais.

O padrão MVP foi implementado usando a variação chamada Passive View onde o
Model fica totalmente isolado da View. Esta variante foi usada pois evitou-se
fazer alterações em classes correspondentes a camada de Model, devido ao alto
acoplamento e grande complexidade do aplicativo, dessa forma foi possível evitar
que a refatoração gerasse algum erro no aplicativo, mantendo-o funcional. A
maioria das altreções estão presentes nas classes criadas para exercerem o papel
de Presenter.

As classes de View selecionadas para a refatoração extendem a classe Fragment.
A classe Fragment é usada para criar agrupamentos de componentes visuais que
possam ser reutilizados em outras partes do aplicativo. O uso da classe
Fragment foi mantido e parte das implementações presentes nessas classes foram
delegadas para as classes de Presenter correspondentes.

As métricas descritas por \citeonline{cksuite} foram usadas para avaliar os
impactos na qualidade do aplicativo após a refatoração usanddo o padrão MVP. Foi
usado o projeto CKJM para coletar os dados para fazer a análise.

Conforme um software é desenvolvido, novas classes e troca de mensagens são
implementadas afetando negativamente as métricas WMC, RFC e CBO. Não é possível
relacionar diretamente a métrica CBO com base na variação dos resultados para
esta métrica, apesar de que, os valore se mantiveram abaixo dos encontrados da
versão de referência, isso requer mais estudos abordando outros padrões de projeto.

%OK
A maioria das métricas tendem a aumentar durante o processo de refatoração. 
É válido resaltar que as três classes refatoradas são as mais complexas do
pacote de grupos com valores anômalos para as métricas. Portanto, somente após
uma refatoração em uma amostragem maior de classes que se é possível determinar
um valor de referência para as métricas, inclusive para um projeto construído
do zero. Com os dados coletados durante o desenvolvimento é possível definir
esses valores limites para identificar anomalias nas classes e tratar cada caso
isolado.

 
\section{Trabalhos Futuros}

Existem outros padrões de projetos para o desenvolvimento da camada de
apresentação de um software que não foram analisados nesse trabalho, a saber: 
MVVM, MVP-VM, MVPC. Essas variações no padrão MVC surgiram em contextos
diversificados e podem agregar algum benefício à qualidade do aplicativo.
Este trabalho não aborda o impacto do padrão MVP em outras métricas de qualidade
de código derivadas do CK.
Não foi feita uma avaliação dos impactos na performance do aplicativo devido ao
uso do padrão MVP. A inclusão de mais objetos interagindo trocando mensagens
pode depreciar a performance, levando-se em consideração sua execução em
ambientes mais restritos, como no caso de um aparelho móvel.