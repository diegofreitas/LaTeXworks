
\if

Como DP resolvem problemas, seguir linha de raciocinio para aplicar padrões em
android.

Referenciar DIT criticando as heranças das classes da api android


Analise do componente Activity, UI Thread, hieráriquia de herança e algumas
funcionalidades que dependem da activity e como isso interfere na aplicação do
padrão(Por experiência confirmo que é negativa). mostrar exemplo de uma view em
lista para smartphone e outra para tablet com grid usando o mesmo modelo

Experimentos


activity - controller+async task - observer pattern,
activity - controller+async task - localbroadcast



\citeonline{Reenskaug:1979} The View and Controller roles may be played by the 
same object when they are very tightly coupled. Example: A Menu., porém isso
requer um boa análise do problema em questão para decidir o nível de
granularidade que esses componentes podem ter portanto é recomendável manter
sempre essa separação.É aplicável para 

Com base na literatura revisada esta seção tem como objetivo fazer uma análise
dos componentes do framework android e projetar uma camada de apresentação
utilizando o padrão MVP. Será realizado uma análise  tendo como objetivo definir
usada um referência de implemntação a ser aplicada na refatoração(Referências
sobre Refatoração de Código)
Quem vai ser o model?

\begin{center}
\begin{tabular}{ | l | l | l | }
  \hline                        
  	Model & View & Presenter \\  \hline
  	Dao & Activity & POJO \\  \hline
\end{tabular}
\end{center}


OnTouchListener pode exercer papel de controller pois
pode ser usado para interpretar os gestos do usuário e direcioar para o model.

evitar que outras classes da camada abaixo dependam das classe de view do
android.

%Explicar uso do padrão observer no mvc
Na definição clássica do padrão MVC é possível identificar a implementação do
padrão Observer usado para a comunicação entre o model e a view.

Observer=Lapsed listener problem=memory leak e bugs porque uma activity pode ser
ou estar sendo referenciada por um listener que se não for desregistrado vai
causar problema

Destacar problema com o observer pattern e pra solucionar isso
usar publish–subscribe messaging pattern

O Broadcast Receiver é uma implementação do padrão pubsub onde diversos
subscribers se registram para receber mensagens(intents) de seu interesse.


Padrões Criacionais. O objeto Application é um singleton, a activity e
fragmentos são criados pelo android e fica dificil aplicar padrões criacionais é
possível identificar o padrão templete method nos ciclos de vida desses
componentes.



 Para concluir a seção sobre MVC e MVP destacar que A principal características
 desses padrões de projetos é que ele promover maior coesão nos citar Tom deMarco 

baseado em outras pesquisa será aplicado padões de projetos  para melhorar a
 qualidade pegar referências,  relacionar Coesão = Qualidade.

\section{Arquitetura da prova de conceito}

Baseado no que foi explicado acima elaborar arquitura para implementar e
apliicar as medições.

Descrever 2 arquiteturas para implemntar, usando uml (referenciar livro de
arquitetura de software)
\fi
\chapter{Resultados e Conclusões}

\section{Conclusões}

É um equívoco MVC no contexto atual pois o padrão não se adequa. o que existe
são evoluções aplicadas a uma determinada situação desenvolvimento web, desktop
e 

abordar quais seriam os parâmetros aceitáveis para as métricas

Validade dessas métricas, talvez o ideal fosse customizar essas métricas para o
framework.(question mark);
Possível aumento do código com essa divisão com mais classes

quais metricas foram afetadas.

\section{Trabalhos Futuros}

Existem outros padrões de projetos para o desenvolvimento da camada de
apresentação de um software que não foram analisados nesse trabalho: MVVM,
MVP-VM, MVPC.

Este trabalho não fez uma avaliação dos impactos na performance do aplicativo
devido ao uso desses padrões. A inclusão de mais objetos interagindo,
redirecionando mensagens pode depreciar a performance.
