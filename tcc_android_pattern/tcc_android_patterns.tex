%% abtex2-modelo-trabalho-academico.tex, v-1.7.1 laurocesar
%% Copyright 2012-2013 by abnTeX2 group at http://abntex2.googlecode.com/ 
%%
%% This work may be distributed and/or modified under the
%% conditions of the LaTeX Project Public License, either version 1.3
%% of this license or (at your option) any later version.
%% The latest version of this license is in
%%   http://www.latex-project.org/lppl.txt
%% and version 1.3 or later is part of all distributions of LaTeX
%% version 2005/12/01 or later.
%%
%% This work has the LPPL maintenance status `maintained'.
%% 
%% The Current Maintainer of this work is the abnTeX2 team, led
%% by Lauro César Araujo. Further information are available on 
%% http://abntex2.googlecode.com/
%%
%% This work consists of the files abntex2-modelo-trabalho-academico.tex,
%% abntex2-modelo-include-comandos and abntex2-modelo-references.bib
%%

% ------------------------------------------------------------------------
% ------------------------------------------------------------------------
% abnTeX2: Modelo de Trabalho Academico (tese de doutorado, dissertacao de
% mestrado e trabalhos monograficos em geral) em conformidade com 
% ABNT NBR 14724:2011: Informacao e documentacao - Trabalhos academicos -
% Apresentacao
% ------------------------------------------------------------------------
% ------------------------------------------------------------------------

\documentclass[
	% -- opções da classe memoir --
	12pt,				% tamanho da fonte
	openright,			% capítulos começam em pág ímpar (insere página vazia caso preciso)
	twoside,			% para impressão em verso e anverso. Oposto a oneside
	a4paper,			% tamanho do papel. 
	% -- opções da classe abntex2 --
	%chapter=TITLE,		% títulos de capítulos convertidos em letras maiúsculas
	%section=TITLE,		% títulos de seções convertidos em letras maiúsculas
	%subsection=TITLE,	% títulos de subseções convertidos em letras maiúsculas
	%subsubsection=TITLE,% títulos de subsubseções convertidos em letras maiúsculas
	% -- opções do pacote babel --
	english,			% idioma adicional para hifenização
	french,				% idioma adicional para hifenização
	spanish,			% idioma adicional para hifenização
	brazil,				% o último idioma é o principal do documento
	]{abntex2}


% ---
% PACOTES
% ---

% ---
% Pacotes fundamentais 
% ---
\usepackage{cmap}				% Mapear caracteres especiais no PDF
\usepackage{lmodern}			% Usa a fonte Latin Modern			
\usepackage[T1]{fontenc}		% Selecao de codigos de fonte.
\usepackage[utf8]{inputenc}		% Codificacao do documento (conversão automática dos acentos)
\usepackage{lastpage}			% Usado pela Ficha catalográfica
\usepackage{indentfirst}		% Indenta o primeiro parágrafo de cada seção.
\usepackage{color}				% Controle das cores
\usepackage{graphicx}			% Inclusão de gráficos
% ---
\usepackage{comment}	
\usepackage{url}
% ---
% Pacotes adicionais, usados apenas no âmbito do Modelo Canônico do abnteX2
% ---
\usepackage{lipsum}				% para geração de dummy text
% ---

% ---
% Pacotes de citações
% ---
\usepackage[brazilian,hyperpageref]{backref}	 % Paginas com as citações na bibl
\usepackage[alf]{abntex2cite}	% Citações padrão ABNT

\usepackage[brazil]{babel}



% --- 
% CONFIGURAÇÕES DE PACOTES
% --- 

% ---
% Configurações do pacote backref
% Usado sem a opção hyperpageref de backref
\renewcommand{\backrefpagesname}{Citado na(s) página(s):~}
% Texto padrão antes do número das páginas
\renewcommand{\backref}{}
% Define os textos da citação
\renewcommand*{\backrefalt}[4]{
	\ifcase #1 %
		Nenhuma citação no texto.%
	\or
		Citado na página #2.%
	\else
		Citado #1 vezes nas páginas #2.%
	\fi}%
% ---


% ---
% Informações de dados para CAPA e FOLHA DE ROSTO
% ---
\titulo{Uso de Padrões de Projetos no desenvolvimento de aplicativos Android}
\autor{Diego Lins de Freitas}
\local{Manaus-Am}
\data{2013, v0.6}
%\orientador{Lauro César Araujo}
\instituicao{%
  Faculdade Fucapi
  \par
  Especialização em Engenharia de Software
  \par
  Centro de Pós-Graduação e Extensão (CPGE)}
\tipotrabalho{Tese (Doutorado)}
% O preambulo deve conter o tipo do trabalho, o objetivo, 
% o nome da instituição e a área de concentração 
\preambulo{}
% ---


% ---
% Configurações de aparência do PDF final

% alterando o aspecto da cor azul
\definecolor{blue}{RGB}{41,5,195}

% informações do PDF
\makeatletter
\hypersetup{
     	%pagebackref=true,
		pdftitle={\@title}, 
		pdfauthor={\@author},
    	pdfsubject={\imprimirpreambulo},
	    pdfcreator={LaTeX with abnTeX2},
		pdfkeywords={abnt}{latex}{abntex}{abntex2}{trabalho acadêmico}, 
		colorlinks=true,       		% false: boxed links; true: colored links
    	linkcolor=blue,          	% color of internal links
    	citecolor=blue,        		% color of links to bibliography
    	filecolor=magenta,      		% color of file links
		urlcolor=blue,
		bookmarksdepth=4
}
\makeatother
% --- 

% --- 
% Espaçamentos entre linhas e parágrafos 
% --- 

% O tamanho do parágrafo é dado por:
\setlength{\parindent}{1.3cm}

% Controle do espaçamento entre um parágrafo e outro:
\setlength{\parskip}{0.2cm}  % tente também \onelineskip

% ---
% compila o indice
% ---
\makeindex
% ---

% ----
% Início do documento
% ----
\begin{document}

% Retira espaço extra obsoleto entre as frases.
\frenchspacing 

% ----------------------------------------------------------
% ELEMENTOS PRÉ-TEXTUAIS
% ----------------------------------------------------------
% \pretextual

% ---
% Capa
% ---
%\imprimircapa

\imprimirfolhaderosto*

% ---
% RESUMOS
% ---

% resumo em português
\begin{resumo}
A plataforma android foi adotada por vários fornecedores de smartphones e
tablets promovendo uma disseminação do sistema operacional muito abrangente.
Formou-se um grande mercado a ser explorado com fornecimento de aplicativos para
as  mais variadas finalidades. Para atender essa demanda é necessário
desenvolver aplicativos de qualidade e o mais rápido possível. Porém, por
se tratar de pequenos  aplicativos há pouca preocupação com a arquitetura. 
Os padrões difundidos na comunidade são focados na utilização de
componentes e desenvolvimento da interação homem-máquina. Neste trabalho são
apresentados alguns padrões de projetos  aplicados ao desenvolvimento de aplicativos android dando
um visão mais ampla de um aplicativo.

 \vspace{\onelineskip}
    
 \noindent
 \textbf{Palavras-chaves}: Android. MVC. MVP.
\end{resumo}

% ---
% inserir o sumario
% ---
\pdfbookmark[0]{\contentsname}{toc}
\tableofcontents*
\cleardoublepage
% ---

% ----------------------------------------------------------
% ELEMENTOS TEXTUAIS
% ----------------------------------------------------------
\textual

\chapter{Introdução}

%Contextualização android vai dominar precisamso programar pra ele
A tendência do mercado de dispositivos móveis é aumentar. Segundo IDC,  é
experado um crescimento de 32.7\% na produção em 2013 \cite{idc:a}. O sistema
operacional android é o líder dominando 75\% desse mercado com mais de 162 milhões de
smartphones produzidos e embarcados com android\cite{idc:b}. Baseado nesses
dados é possível concluir que existe uma damanda no desenvolvimento de novos 
aplicativos que agregem valor à esses aparelhos.
% é preciso que tenha   - Design Principles.

Para produzir aplicativos de qualidade é necessário aplicar boas práticas de
desenvolvimento de software. Levando-se em consideração que a linguagem de
programação usada para o desenvolvimento na plataforma android é a Java, é
natural que se aplique as práticas definidas pelos princípios de projeto
orientado a objetos. Esses princípios guiam o desenvolvedor em como definir a
estrutura interna do software afetando diretamente características  de
qualidade como manutenabilidade, performance e outras\cite{tempero-di}.

Os padrões de projeto são aplicados na engenharia de software como forma de
reproduzir  soluções  para problemas recorrentes melhorando a manutenabilidade e
o reuso de componentes de software\cite{gof}.O uso de padrões de projetos
é um meio de aplicar esses princípios. 

Boas práticas de engenharia de software promevem o desenvolvimento dos
artefatos que constituem o sistema de forma coesa. Os conjuntos desses artefatos
podem ser classificados de acordo com suas responsabilidades, surgindo
uma divisão clara em camadas. Um sistema pode ter diversas camadas, responsáveis
por acesso à um banco de dados, comunicação com serviços externos, encapsular
regras de negócio. Uma dessas camadas que constituem um aplicativo android é a
camada de apresentação ou interface com o usuário.
 
Segundo \citeonline{pressman}, ``A interface com o usuário pode ser considerada
o elemento mais importante de um sistema ou produto baseado em computador``.
Tendo em vista isso, será tratado nesse trabalho  os padrões para
desenvolvimento da camada de apresentação de em aplicativos android.

\section{Motivação}

A motivação para a execução desta pesquisa sugiu da participação do autor em
projetos para desenvolvimento de aplicativos android em uma instituição de
pesquisa e desenvolvimento localizada em Manaus. O comprometimento com a
qualidade promoveu o uso de ferramentas de código aberto para monitoração da
qualidade dos projetos, além do processo de testes adotado pela empresa. Com uma
equipe composta por mais de 30 desenvolvedores produzindo aplicativos com
diversas finalidades, houve a necessidades de padronização da arquitetura.

\section{Problematização}
Dado que existem uma vasta diversidade de padrões de projeto a serem aplicados
no desenvolvimento de software, Qual padrão de projeto é aplicável para o
desenvolvimento da camada de apresentação de aplicativos android para melhorar a
qualidade do produto final?

\section{Hipótese}

O padrão de projeto Model-View-Presenter é aplicável no desenvolvimento da
camada de apresentação de aplicativos android, gerando um aumento da qualidade
do produto final.

\section{Objetivos}

Este trabalho fará um estudo sobre a arquitetura de aplicativos android com o
propósito de melhorar a qualidade desses softwares, causando uma redução nos
riscos relacionados as mudanças que acontecem durante o ciclo de desenvolvimento
e promova a reutilização de componentes e que permita o incremento de funcionalidades com
baixo impacto. Este estudo fornecerá insumos para que os desenvolvedores possam
ter uma visão geral da aplicabilidade dos padrões de projetos e analisar quais
técnicas contribuem para o seus projetos. Este trabalho tem como meta:

\begin{itemize}
\item Estudar os padrões de projeto que podem ser usados para a implementação da
camada de apresentação de um aplicativo android.
\item Avaliar os componentes do  framework android, identificando seus papéis e
responsabilidades de acordo com os padrões estudados, levando  em consideração
características que podem dificultar a aplicação dos padrões. 
\item Identificar os impactos nas características de qualidade do aplicativo de acordo
com o paradigma da Orientação a objetos.
\item Propor um referência de implementação para a  camada de apresentação de um
aplicativo android utilizando padrões de projetos.
\end{itemize}

\section{Trabalhos Relacionados}

Na dissertação de \citeonline{turk} é feita uma análise dos impactos na
qualidade ao aplicar cinco padrões de projetos em um software de comunicação
TCP/IP. Os resultados do trabalho citado mostram que a qualidade do software
aumenta ao aplicar padrões de projetos sendo que o principal atributo de
qualidade aferido é a manutenabilidade. O método utilizado para a execução da
pesquisa se baseia no trabalho citado. 

Não foi encontrado na literatura disponível, referências sobre a aplicação do
MVP para desenvolvimento de aplicativos android. Outros trabalhos mostram a
aplicação do padrão em outros contextos e plataformas  \cite{presenterfirst},
\cite{yangmvp}. Esses trabalhos enfatizam que o MVP melhora a
testabilidade, que é uma característica de qualidade de software. Outra
característica subjetiva da qualidade de software é a manutenabilidade.
\citeonline{Dubey:2011} sumarizam uma série de estudos que mostram a relação
entre sistemas orientados a objetos e sua manutenabilidades, e como as métricas
de \citeonline{cksuite} são eficazes para descrever de forma quantitativa essa
relação.

\section{Contribuições}


Tendo em vista que padrões de projetos descrevem soluções genéricas, este
trabalho tem como principal contribuição uma interpretação do padrão de projeto
Model-View-Presenter, proporcionando uma referência prática aplicada ao
framework android.

A diversidade de projetos de software torna difícil inferir se as métricas
podem ser consideradas um paramêtro para determinar a qualidade. O presente
trabalho  se propõe a avaliar a aderência do padrão MVP à projetos android
através de indicadores de qualidade de código orientado a objetos.

\section{Metodologia}

Esta pesquisa se caracteriza como experimental pois é selecionado um objeto
de estudo que será analisado do ponto de vista de métricas bem definidas. Essas
métricas serão influenciadas pelos experimentos de refatoração embasados na
revisão literatura existente sobre orientação a objetos e seus princípios.

Para validar a hipótese apresentada neste trabalho a pesquisa terá uma
abordagem quantitativa através da análise de dados estatistícos de métricas 
que expresssam atributos de qualidade em software orientado a objetos. O
conjunto de métricas a serem usadas nessa validação será o elaborado por
\citeonline{cksuite}.

Essas medidas serão coletadas através de um procedimento experimental em
laboratório utilizando um processo iterativo-incremental para executar
refatorações no código do objeto de estudo afim de introduzir o padrão de
projeto Model-View-Presenter e a cada iteração será feita a coleta das
métricas. Para executar esse processo será identificado um conjunto de
funcionalidades que o aplicativo atende, relacionda com a camada de
apresentação com a qual o usuário interage.

\subsection{Objeto de Estudo}


O projeto a ser refatorado será o aplicativo de Contatos do
android\footnote{\url{https://android.googlesource.com/platform/packages/apps/Contacts}},
que é um dos aplicativos básicos pré-instalados com o sistema operacional. Este
projeto é opensource mantido pelo  The Android Open Source Projeto suportado
pela Google com contribuições de desenvolvedores do mundo todo.

Os critérios para a escolha do objeto de estudo são:

\begin{enumerate}
  \item Tamanho/Complexidade - Um projeto muito complexo iria inviabilizar a
  pesquisa devido ao esforço para fazer a refatoração. Métricas coletadas
  apartir de projetos simples e triviais não forneceriam dados suficientes para
  uma análise satisfatória.
  \item Código Aberto - Além de permitir o acesso ao código fonte sem
  limitações para a pesquisa, tem a contribuição de vários desenvolvedores com
  experiência e formação em programação diversificada que se refletem no código fonte.
  \item Origem do projeto - A escolha de um aplicativo mantido sobre o mesmo
  gerenciamento que o sistema operacional android foi feita com o intuito de
  fazer a pesquisa em um código fonte que expressasse as técnicas e práticas de
  programação difundidas nesse ecossistema.
\end{enumerate}

\subsection{Processo de Experimentos}


A Figura \ref{processo_experimentacao} demonstra as atividades do processo de
experimentação adotado:
\begin{figure}[!h]
	\centering
	\includegraphics[scale=0.4]{img/processo_experimentacao.png}
	\caption{Processo de Experimentação Fonte: Próprio Autor}
	\label{processo_experimentacao}
\end{figure}

\begin{description}
\item[Definir versão de referência] - Delimitar um marco do estado do código no
repositório.
\item[Refatoração de funcionalidade] Esta atividade tem como objetivo aplicar os
padrões propostos em uma funcionalidade do aplicativo.
\item[Construir proejto] Excutar o processo de construção do projeto que inclui
a compilação das classes para fazer a coleta das métricas.
\item[Coleta de Métricas] Este passo tem como objetivo fazer a coleta
das métricas do código que se encontra no repositório a partir de uma revisão
para fazer a avaliação dos efeitos da refatoração na qualidade do código.
\item[Análise dos resultados] Discussão dos impactos das alterações executadas
nas métricas.
\end{description}


\section{Organização do Trabalho}

O presente trabalho está estruturado em 4 capítulos dos quais este é o
Capítulo 1 que apresentou a proposta do trabalho, contexto do problema, a
motivação para estes estudo, os objetivos a serem alcançados e a estrutura do
trabalho. Também é abordado a metodologia aplicada neste trabalho mostrando o
objeto de estudo a ser analisado e os passos a serem executados, enfatizando o
método experimental.
O Capítulo 2 discorre sobre a fundamentação teórica sobre padrões de projetos,
métricas de qualidade orientada a objetos e as tecnologias com as quais o objeto
de estudo foi desenvolvido.
O Capítulo 3 contém uma análise do objeto de estudo e define como serão feitas
as implementações e mostra os resultados da execução da pesquisa.
O Capítulo 4 faz a análise final, conclusões do trabalho, e sugestões para
trabalhos futuros.


\chapter{Referencial Teórico}

\section{Princípios e Padrões de Projetos}


Desenvolver software orientado a objetos é um desafio. Criar uma representação
computacional de uma faceta da realidade em que seus constituintes trabalhem de
forma harmoniosa para atingir as necessidades que o software se propõe a
atender requer experiência, conhecimento do domínio do problema e um processo de
análise e projeto. Apesar de existir várias abordagens para se conceber um
sistema orientado a objetos\cite{evans2004ddd},\cite{gomma11} um sistema bem
contruído apreseta características fundamentais como alta coesão e
baixo acoplamento.

A Coesão é uma característica de um componete de software que se refere ao grau
de relacionamento entres os menbros desse componente. No contexto de uma classe,
é levado em consideração as relações entre os métodos e atributos. Classes com
coesão baixa demonstram grande complexidade pois os atributos e métodos que não
se relacionam indicam que a classe tem muitas responsablidades.

O acoplamento descreve as dependências entre componetes. Quanto maior a
quantidade dessas dependências entre classes, mais complexo ela se torna, pois
dificulta  alterações na classe. Além disso, o acoplamento aumenta o risco de
uma classe ser afetada devido à alterações em suas dependências.

Um padrão, dentro do contexto de estudo deste trabalho pode ser definido
como uma técnica efetiva cuja a sua aplicabilidade é aceita e difundiida dentro
de uma área de conhecimento com a intenção de atingir um
objetivo\cite{MetskerWake06}.

Em desenvolvimento de software, o catálogo mais difundido de padrões de projetos
orientado a objetos é elaborado por \citeonline{gof}, contendo um total de 23
padrões formalmente documentados que acumulam experiências bem sucedidas em
diversos sistemas. Esses padrões têm a seguinte classificação:

\begin{description}
\item[Criacionais] Padrões que definem como criar novas instâncias de classes.
\item[Estruturais] Foca na estruturação das classes e objetos.
\item[Comportamentais] Definem como as classes e objetos interagem entre si e
suas responsabilidades.
\end{description}

%argumentação
Analisando a forma como um padrão de projeto é concebido, com uma definição
dos papéis de cada elemento participante e como eles interagem entre si,
pode-se concluir que o uso dos padrões de projeto promove maior coesão, melhor
separação de interesses e baixo acoplamento no sistema. Todas essas
características são muito importantes e contribuem para um software de melhor
qualidade.


%Melhorar isso

%Linkar com as metricas, mostrar relação metricas e padroes

\section{Métricas de qualidade OO}
\label{sec:metrics}

Com o advento de novas técnicas de desenvolvimento de software é necessário
obter informações do impacto dessas inovações nos resultados de um projeto. Com
esse objetivo \citeonline{cksuite} elaboram um conjunto de métricas para
mensurar a qualidade de sistemas desenvolvido usando o paradigma orientado a objetos que
não se limitasse a uma linguagem de programação, fácil de coletar e com forte
embasamento teórico na ontologia de bunge\footnote{}. This model is used to analyze
some static and dynamic properties of an information system and to examine the
question of what constitutes a good decomposition of an information
system\cite{WandWeber}As métricas são:

%Bunge’s ontology has
%considerable appeal for 00 researchers since it deals with the
%meaning and definition of representations of the world, which
%are precisely the goals of the object oriented approach [32]



\begin{description}
\item[Acoplamento entre objetos (CBO)] Número de classes que ela depende por
meio da relação de composição. Uma classe está acoplada a outra quando o método
de um classe invoca o método de uma variável de instância de outra classe que
gera uma dependência entre essas classes. Quanto maior essa dependência, mais
difícil é reutilizar esses componetes em outras partes do sistema, além do
aumento do risco de efeitos colaterais ocorrerem ao modificar uma classe
altamente acoplada.
\item[Ausência de coesão dos métodos(LCOM)] Usado para avaliar a coesão de uma
classe através da similaridade entre seus métodos. Um método tem similaridade
com outro quando a intercessão entre os conjuntos de atributos usados por ambos
os métodos tem cardinalidade maior que zero. LCOM mostra esses conjuntos nulos
indicando os métodos que usam esses atributos deveriam ser implementados em
outra classe. Essa similaridade expressa a coesão da classe.
\item[Profundidade na árvore de herança (DIT)] Nível de uma classe na
hierárquia de herança. Reflete o número máximo de elementos pai dentro da aŕvove
de classes até a raiz, o que aumenta a complexidade conforme a quantidade de
elementos envolvidos se eleva, diminuindo a previsibilidade do comportamento da
classe com vários métodos e atributos sendo herdados, principalmente com o uso
de sobrecarga de métodos.
\item[Métodos por Classe (WMC)] Serve para expressar o nível de complexidade de
uma classe basseado no número de métodos que ela possui. Isso afeta o esforço de
manutenção da classe, além de impactar nas classe filhas que herdarão esses
métodos. Também é um indicativo de que a classe tem métodos específicos
dificultando o seu reuso.
\item[Número de classes filhas (NOC)] Número de subclasses imediatas de uma
classe. Essa medida é um indicativo de mau uso de herança conforme seu valor
aumenta e mostra o impacto que uma classe pode ter no sistema requerendo maior
atenção e testes.
\item[Response sets for Class (RFC)] Quantidade de métodos que são executados
quando um objeto recebe uma mensagem, incluindo os métodos de outras classes. 
\end{description}

Todas essas métricas tem uma relação inerente com a coesão e acoplamento dos
objetos, sendo uma forma confiável para a análise da qualidade em sistemas
orientados a objetos. Os aplicativos desenvolvidos para a plataforma android são
escritos usando a linguagem de programação Java que emprega esse paradigma de
desenvolvimento, o que justifica o uso das métricas de \citeonline{cksuite} para
validação dos projetos orientados a objetos.

\section{Model View Controller}

O padrão Model View Controller surgiu como uma solução genérica para que
usuários de uma sistema de planejamento manipulem dados complexos
\citeonline{Reenskaug:1979}. Posteriormente, \citeonline{krasnerPope1988}
implementam um framework MVC para o ambiente gráfico da linguagem de programação
Smalltalk-80 como uma forma de promover a reusabilidade e plugabilidade.

Segundo \citeonline{Reenskaug:1979} o principal objetivo do MVC
``\ldots é representar o modelo mental do usuário de um espaço de informações
relevantes e permitir que o usuário inspecione e altere esta
informação.''(tradução livre).
Esse modelo mental é como o usuário percebe o dominio do problema que está inserido no qual executará suas atividades sobre dados de seu interesse. Para que o usuário de um sistema de
informação possa interagir com a represetação computacional  de seu modelo
mental três componetes são definidos:

Models - É o compoente constituído de uma composição de classes que implementam
as regras de negócio referentes as funcionalidades que o programa provê,
representa o  conhecimento que o usuário tem e como manipula-lo. Atende
mensagens da view requisitando seu estado e mensagens do controller para mudar
seu estado,

Views - Representação específica de um model na interface com o usuário, é 
responsável por todo a manipulação visual, recuperando um estado do model e
exibindo os dados, podendo ser composta por sub-views e ser parte de views mais
complexas.

Controllers - Interpreta a as ações do usuário provenientes de um dispositivo de
entrada(Teclado, Mouse) alterando estado da view ou do model.




\citeonline{krasnerPope1988} descrevem a estrutura do MVC onde a view tem seu
controller exclusivo mantendo uma dependência cíclica entre ambos. Tanto a View
quanto o Controller tem referências diretas para o model por meio de atributos
de classe, porém, o model não deve conhecer seus respectivos pares de
View-Controller para promover maior reuso de código e encapsulamento do model.
As alterações do estado do model são feitas na maioria das vezes pelo controller, e
o model é responsável por notificar todas as views que o representa para que
se atualizem refletindo o novo estado. No caso de um model ser usado por vários
pares de View-Controller as mensagens de notificação de um novo estado do model
podem ser parametrizadas assim cada view pode verificar se a alteração é de seu
interesse. 

Segundo \citeonline{Fowler:2002:PEA} ``\ldots esta separação da
apresentação e modelo é uma das mais fundamentais herísticas de bom projeto
de software''(tradução livre).
O controller poderia ser o responsável por publicar as alterações no estado do
model devido sua relação direta com o mesmo, mas em casos onde o model é
alterado por outro componente que não é um dos controladores que os utilizam, é
necessário que o model conheça as views que devem ser notificados do novo
estado. Para que essas alterações de estado sejam propagadas a view e o
controller são registrados como dependentes de seu model. O padrão é descrito
dentro do contexto no qual o surgiu levando em consideração caracteristicas
espefíficas da  linguagem de programação que dão suporte à implementação dos
três componentes como por exemplo o gerenciamento dos objetos que são
dependentes do model definido na classe Objet que o model deve extender. A
Figura \ref{mvc_seq} esclarece a interação entre os componetes.

\begin{figure}[h]
	\centering
	\includegraphics[scale=0.5]{img/mvc_seq.png}
	\caption{Diagrama de Sequência do MVC/Fonte: Próprio Autor}
	\label{mvc_seq}
\end{figure}

\citeonline{gof} cita \citeonline{krasnerPope1988} fazendo uma análise dos
objetos que compõem o MVC relacionando-os com outros padrões de proejto
descritos em seu catálogo.
O desacoplamento entre a View e o Model, somado à propagação das mudanças de
estado no model para os objetos regitrados como dependentes do model pode ser
descrito como uma implementação do padrão Observer. O padrão Observer define uma
estrutura em que um componente que precisa publicar mudanças em seu estado, detém a
referência para uma lista de objetos a serem notificados. A hierarquia de views
é um exemplos de Composite pois uma view pode ser constituída por sub-views para
compor views complexas. No Composite  um conjunto de componentes podem ser
tratadas de forma encapsulada onde cada implementa as mesmas abstração. O padrão
Strategy define uma abstração cuja as implementações podem ser trocadas de
acordo com algum critério, esse conceito pode ser aplicado ao controller que
encapsula o algoritmo que vai alterar a View e o Model, permitindo sua
substituíção por uma outra implementação que deixa de responder às interações
com o usuário.

Segundo \citeonline{krasnerPope1988} o Model ``\ldots pode ser simples como
um valor numério inteiro (como o modelo de um contador) ou um valor literal
(como o modelo de um editor de texto), ou pode ser um objeto
complexo''(tradução livre).
O model pode ser implementado usando o pardrão Facade para simplificar as
interações com o Model dependendo da complexidade do domínio que ele
representa.

\section{Model View Presenter}

O MVP é um modelo de programação para implementação de interfaces com o usuário
desenvolvido como um framework para C++ e Java, criado por uma subsidiária da
IBM chamda Taligent,Inc. Este padrão é baseado no MVC e descreve vários componentes que tem as
responsabilidades de como gerenciar os dados da aplicação e como o usuário
interage com esses dados, tendo como objetivo promover o encapsulamento do Model
, reuso de lógica de negócio e o polimorfismo da View.

\begin{description}
  \item[Model] Tem as mesmas responsabilidades que o Model definido pelo MVC.
  \item[Selections] - Abstração para selecionar um subconjunto dos dados
  existentes no model.
  \item [Commands] Representa as operações a serem executadas sobre uma
  Selection do Model.
  \item [View] Responsável por exibir o model assim como no MVC.
  \item [Interactor] Mapeia os interações do usuário na view como eventos do
  mouse.
  \item [Presenter] O papel do presenter é interpretar o eventos iniciados pelo
  usuário executando a lógica de negócio correspondente implementada em um
  command para manipular o model \cite{Potel96mvp}.
\end{description}


Os conceitos do MVP são descritos em \citeonline{Potel96mvp} de forma genérica
permintindo interpretações para uma implementação efetiva.
\citeonline{twisttriad:2000} descreve a implementação de um framework para
Dolphin Smalltalk\footnote{Implementação da Linguagem de programação Smalltalk - 
\url{http://www.object-arts.com}} adotando os conceitos do MVP onde salienta que
a maioria dos sistemas operacionais com ambiente gráfico fornece um conjunto de
componentes (Widgets) no qual está contido a responsabilidade do controller.A
maior parte do comportamento do caso com o usuário é implementada no
Presenter que está diretamente associado à View.

Ainda acerca das responsabilidades do Presenter, \citeonline{fowler:ui} descreve
o que é chamado de Passive View, onde toda a lógica do comportamento da view é
implementado no presenter deixando a view enxuta com o intuito de isolar ao
máximo a API gráfica do resto da aplicação. Dessa forma o model não se comunica
com a view por meio do observer pattern, sendo que a view séra atualizada pelo
presenter como pode ser observado na Figura~\ref{fig:mvp_passive_view}.

\begin{figure}[h]
	\centering
	\includegraphics[scale=0.5]{img/mvp_passive_view.png}
	\caption{Passive View/Fonte: Próprio autor}
	\label{fig:mvp_passive_view}
\end{figure}

MVP se adequa melhor as apis gráficas existentes e define de forma mais clara os
componetes necessários para desenvolver uma aplicação, sendo o ponto de maior
discussão reside em quais os limites das responsabilidades no que tange a
mediação do Model e a View por parte do Presenter.

\section{Framework Android}
 

O android é um sistema operacional baseado no linux mantido pela Google para
ser embarcado em dispositivos podendo ser aplicado em carros, televisão, placas
controladoras mas seu destaque é a utilização em smartphones e
tablets, que é o foco deste trabalho. A plataforma é contituída por API's e
frameworks tendo em sua base o sistema operacional e seus drivers seguido da
máquina virtual que executa os aplicativos android e bibliotecas auxiliares e
aplicativos básicos como é demonstrado na figura \ref{android_stack}.

\begin{figure}[h]
	\centering
	\includegraphics[scale=0.5]{img/android_stack.png}
	\caption{Android Stack/Fonte: Learning Android}
	\label{android_stack}
\end{figure}

Para desenvolvimento é usado a api disponível no sdk que define
os blocos de construção de um aplicativo, a saber:

\begin{description}
  \item[Activity] Representa uma atividade que o usuário executa no aplicativo
  em um determinado momento. É um agregador de componetes visuais e responde à
  interações do usuário.
  \item[Fragment] Representa uma parte de interface com o usuário em uma
  Activity.
  \item[Service] Responsável por executar uma operação sem interface gráfica
  indicado para processamentos longos como por exemplo a execução de uma música
  ou download de de arquivos.
  \item[Broadcast Receiver] Implementação do padrão publish/subscribe 
  \item[Content Provider] Usado para expor dados de uma aplicativo para outros
  aplicativos. Os dados podem ser provenientes de qualquer forma de
  armazenamento como um arquivo ou banco de dados.
  \item[ApplicationContext] Representa a aplicação em execução provendo acesso
  a recursos.
  \item[AsyncTask] Usado para implementar computação paralela evitando o uso da
  linha de execução principal do aplicativo que é respoonsável por tratar a
  interações com o usuário.
\end{description}

Com base nos componentes de framework e literatura revisada é possível fazer
uma análise dos mesmos e projetar uma camada de apresentação utilizando o padrão
MVP para ser usada como referência de implementação a ser aplicada.

\chapter{Execução da Pesquisa}


\section{Ferramentas Aplicadas}

O código será versionado no
Github\footnote{\url{https://github.com/diegofreitas/platform_packages_apps_contacts}}
onde será feito o gerenciamento das versões de cada iteração.
As ferramentas utilizadas para a refatoração serão a IDE Eclipse(Juno) com
plugin ADT v21 para facilitar a edição do código e ferramentas de construção
do projeto existentes no próprio repositório do android, tendo em vista que todo
o processo de compilação e empacotamento não visa ser usado em uma IDE.
Para realizar a coleta das métricas é necessário que a ferramenta analise código
java e contemple todas as métricas descritas na seção \ref{sec:metrics}. O
programa Chidamber and Kemerer Java
Metrics(CKJM)\footnote{\url{https://github.com/dspinellis/ckjm}} atende esses
critérios, além de ser um projeto de código aberto. O CKJM é uma aplicação java
sem interface gráfica, executado por linha de comando. Ele analisa o código java
compilado, conhecido como byte codes contido em arquivos com extensão .class. Um
exemplo de uso do CKJM é mostrado a seguir:

\begin{figure}[htb]
	\caption{\label{fig:ckjm_run} Exemplo de execução do CKJM} 
	\begin{center}
		\includegraphics[scale=0.5]{img/ckjm_run.png}
	\end{center}
	\legend{Fonte: Próprio Autor}
\end{figure}

Os resultados são escritos na saída padrão do sistema, neste caso, no terminal
de execução, mostrando uma lista com os nomes completos das classes
analisadas, seguidas dos respectivos valores das métricas na seguinte
ordem: WMC, DIT, NOC, CBO, RFC, LCOM, Ce, and NPM sendo que as métricas Ce e
NPM são desconsideradas para esta pesquisa. a figura~\ref{fig:ckjm_result}
ilustra o resultado de uma coleta.

\begin{figure}[htb]
	\caption{\label{fig:ckjm_result} Exemplo de resultado da análise do CKJM} 
	\begin{center}
		\includegraphics[scale=0.45]{img/ckjm_result.png}
	\end{center}
	\legend{Fonte: Próprio Autor}
\end{figure}


\section{Análise do objeto de estudo}

O aplicativo a ser refatorado tem funcionalidades para gerenciamento de
contatos e é composto 153 classes organizadas em 13 pacotes. O padrão
de projeto MVP será aplicado em uma parte do aplicativo. Será refatorado o
pacote referente ao gerenciamento de grupos de contatos presente no pacote
\textbf{com.android.contacts.group}. A figura \ref{fig:pacotes_contacts}
mostra os pacotes que compõem o aplicativo. O pacote a ser refatorado está
destacado em azul.

\begin{figure}[htb]
	\caption{\label{fig:pacotes_contacts} Diagrama de pacotes do aplicativo de
	contatos}
	\begin{center}
		\includegraphics[scale=0.40,angle=90]{img/pacotes_contacts.png}
	\end{center}
	\legend{Fonte: Próprio Autor}
\end{figure}

A cada iteração será selecionada uma tela do aplicativo para a aplicação do
padrão MVP. Cada tela é implementada por uma classe e suas interfaces
são ilustradas nas figuras \ref{fig:contacts_groups},
\ref{fig:contacts_groups_view} e \ref{fig:groups_edit}. 

\begin{figure}[htb]
	\centering
	\begin{minipage}[b]{0.45\linewidth}
		\caption{\label{fig:contacts_groups} Tela de lista de grupos} 
		\begin{center}
			\includegraphics[scale=0.18]{img/contacts_groups.png}
		\end{center}
		\legend{Fonte: Próprio Autor}
	\end{minipage}
\quad
	\begin{minipage}[b]{0.45\linewidth}
		\caption{\label{fig:contacts_groups_view} Tela de visualização de um grupo}
		\begin{center}
			\includegraphics[scale=0.18]{img/contacts_group_view.png}
		\end{center}
		\legend{Fonte: Próprio Autor} 
	\end{minipage}
\quad
	\begin{minipage}[b]{0.45\linewidth}
		\caption{\label{fig:groups_edit} Tela de edição e criação de grupos}
		\begin{center} 
			\includegraphics[scale=0.18]{img/contacts_edit.png}
		\end{center}
		\legend{Fonte: Próprio Autor} 
	\end{minipage}
\end{figure}


As classes a serem refatoradas são:
\begin{description}
\item[GroupDetailFragment.java] Exibe os dados de um grupo de contatos.
\item[GroupBrowseListFragment.java] Fornece uma lista de grupos.
\item[GroupEditorFragment.java] Disponibiliza um formulário para edição dos
dados de um grupo.
\end{description}

Essas classes estão presentes no pacote \textbf{com.android.contacts.group} como
mostra a figura \ref{fig:classes_group_baseline}.

\begin{figure}[htb]
	\caption{\label{fig:classes_group_baseline} Diagrama de pacotes de grupos}   
	\begin{center}
		\includegraphics[scale=0.53]{img/classes_group_baseline.png}
	\end{center}
	\legend{Fonte: Próprio Autor}
\end{figure}

Estas interfaces com o usuário são usadas dentro de Activities que controlam uma
parte do fluxo de interação e se comportam de forma diferente conforme o tipo de dispositivo
móvel utilizado (tablet ou smartphone). Devido a essa complexidade, não será
feita nenhuma alteração na interface pública dos componentes refatorados, evitando efeitos colaterais em
outras partes do aplicativo. Os componentes elencados contêm código não somente
relacionado com a lógica de apresentação como também interagem diretamente com classes destinadas ao acesso
de dados e serviços existentes nas dependências do projeto, por exemplo,
gerenciamento de contas do usuário.  Cada iteração consistirá na refatoração de
cada um dos componentes descritos. O marco de referência de dados das métricas presentes na tabela \ref{tab:dados_baseline} será feita a partir da
versão \verb|4.4.2_r1| do aplicativo.

\begin{table}[!h]
	\centering
	    \caption{\label{tab:dados_baseline} Métricas CK da versão de referência}
	
    \begin{tabular}{ | l | l | }
    \hline
    Métrica &	Média \\ \hline
    WMC  	&	8.5161290323   	\\ \hline
    DIT	 	&	0.7741935484	\\ \hline
	NOC  	& 	0				\\ \hline
	CBO	  	& 	10.1612903226	\\ \hline
	RFC	 	& 	23.7419354839	\\ \hline
	LCOM 	& 	57.4838709677	\\ \hline
    \end{tabular}
    \legend{Fonte: Próprio Autor}
\end{table}


Os dados de referência apresentados na tabela \ref{tab:dados_baseline} foram
coletados usando a ferramenta CKJM. Os dados são referentes ao pacote de grupos.
Foram coletados os valores de cada métricas para cada uma das classes presentes
no pacote de grupos e para cada métrica foi calculado a média. A cada
refatoração será executado o mesmo procedimento de coleta descrito para fazer
uma análise na variação das médias de cada métrica.


\section{Arquitetura Proposta}

Será aplicado nos experimentos a variação do padrão MVP chamada Passive View,
pois dessa forma, o Model não precisa publicar alterações de seu estado para a
view, Logo, evita-se alterações no código referente às classes que fazem
parte da camada de Model do aplicativo de contatos. A organização do código
fonte no repositório dificulta a implementação, isto porque esses componentes estão
localizados fora do projeto afetado e são compartilhados.

As classes que extendem Fragment terão a responsabilidade da View, pois é neste
componente que a interface com o usuário é construída. A classe Activity fornece
vários métodos para recuperação de recursos de imagens, textos, inicialização de
serviços, entre outros. Isso ocorre porque a classe Activity é uma subclasse de Context, herdando diversos métodos não relacionados ao gerenciamento da interface.

Segundo \citeonline{Reenskaug:1979} ``\ldots Os papéis da View e do
Controller podem ser exercidos pelo mesmo objeto quando eles estão muito
acoplados. Exemplo: Um Menu.''(tradução livre). Porém, isso requer um boa
análise do problema em questão para decidir o nível de granularidade que esses componentes podem ter.
Portanto, é recomendável manter sempre essa separação para manter uma boa coesão
nas classes. O Presenter será uma classe auxiliar à View e pode ser implementada
como uma classe java simples. 



\section{Resultados dos Experimentos}

Esta seção tem como objetivo mostrar os resultados obtidos com o processo de
refatoração aplicando o parão MVP. Cada métrica é apresentada com seus dados
para cada iteração mostrando os efeitos desses valores na qualidade.

\subsection{WMC}

%1o) qual o cenario ideal para metrica;
A métrica WMC é usada para medir o tempo e esforço necessário para desenvolver e
manter uma classe. Levando em consideração o método como uma unidade de
complexidade, quanto menos métodos um classe tiver menos complexa ela será,
portanto, é recomendado que esta métrica tenha valores baixos.
Entretanto, uma classe terá a quantidade de métodos necessária para exercer seu
papel no sistema. É inviável desenvolver um sistema cujas todas as classes
tenham um único método com a implementação de todas as funcões de uma
classe. Portanto, não existe um valor ideal para a métrica WMC. Essa métrica
deve ser analisada levando em consideração o contexto da classe de interesse além da
complexidade expressa pela quantidade de métodos. 

A possibilidade de reuso de uma classe reduz, pois a grande quantidade de
métodos indica que ela tem funções muito específicas\cite{cksuite}. Neste caso o
WMC é um indicador de que é necessário fazer um refatoração para extrair funções comuns a outras partes do
aplicativo, além do pacote de Groups, como por exemplo funções para exibição de
dados em uma lista, validação de dados em componentes de texto, entre outros. O
escopo de atuação do padrão MVP é mais amplo, abrangendo o caso de uso realizado pela
interface refatorada. Logo, o MVP não tem impacto na reusabilidade.

Os efeitos colaterais em uma classe filha é maior quando é feito alguma
alteração na classe pai que tenha o número de métodos muito alto\cite{cksuite}.
Isso dificulta a manutenabilidade e o esforço de testes. Porém, As classes
refatoradas tem uma função muito específica dentro da aplicação e não são
extendidas por outras classes. O MVP está relacionado com a separação de
responsabilidades e sua aplicação não interferiu na hierarquia de classes que
foram refatoradas.
A tabela \ref{tab:wmc} e a figura \ref{fig:wmc} mostram os valores dessa
métrica no projeto.


\begin{table}[!h]
	\centering
	\caption{\label{tab:wmc}Dados métrica WMC}
    \begin{tabular}{ | l | l | }
    \hline
    Iteração & Média 			\\ \hline
    Baseline & 8.5161290323   	\\ \hline
    Iteração 1 & 8.875			\\ \hline
	Iteração 2 & 9				\\ \hline
	Iteração 3 & 8.9393939394	\\ \hline
    \end{tabular}
    \legend{Fonte: Próprio Autor}

\end{table}


\begin{figure}[!htb]
	
	\caption{\label{fig:wmc} Gráfico da métrica WMC}   
	\begin{center}
		\includegraphics{img/wmc.png}
	\end{center}
	\legend{Fonte: Próprio Autor}
\end{figure}


%3o) justificar o q causou tal comportamento.

De acordo com o que foi exposto, o MVP não deveria afetar a métrica WMC. Porém,
houve um aumento nos valores dessa métrica . Isso é consequência da divisão de
responsabilidades entre a View e o Presenter. Antes da refatoração, os métodos
das classes de View implementavam o acesso a dados, controle dos componentes
visuais e regras específicas das operações executadas na tela, como por exemplo,
a atualização de um componente visual quando nenhum dado está disponível para
exibição. Os métodos da classe de View, afetados pela refatoração, foram
dividídos em no mínimo dois métodos. Sendo que, o método que permanece na View
manipula os componentes visuais, enquanto o novo método criado no Presenter faz
o acesso à camada de Model e implementa as regras de negócio da operação em
resposta à interação do usuário na View. Como consequência a quantidade de
métodos aumenta, porém, os métodos tornam-se mais simples pois implementam
funcões específicas. 

Levando isso em consideração, apesar dos valores de WMC
terem aumentado, a complexidade diminuiu, pois os métodos estão mais concisos.
No contexto dessa refatoração, a métrica WMC não pode ser considerada um
indicador determinante para a avaliação da qualidade do objeto de estudo.


\subsection{NOC}

Altos valores para a métrica NOC são indicativos de que existe maior reuso de
código. Assim como DIT, a métrica NOC é um indicador da influência da classe
no comportamento das classes filhas o que aumenta o esforço de testes. Quando os
valores desta métrica estão aberrantes em relação às outras classes, há grande
chance de que a abstração está sendo usada de forma incorreta. Dado esses cenários, está é um métrica cujos
seus valores devem ser analisados caso a caso.

Durante o processo de refatoração não foi aplicada herança em nenhuma das
classes afetadas, portanto, essa métrica permaneceu intacta durante as iterações
como pode ser observado na tabela \ref{tab:noc}. %e figura~\ref{fig:noc}

\begin{table}[!h]
	\centering
	    \caption{\label{tab:noc} Dados métrica NOC}
    \begin{tabular}{ | l | l | }
    \hline
    Iteração & Média 			\\ \hline
    Baseline & 0  	\\ \hline
    Iteração 1 & 0			\\ \hline
	Iteração 2 & 0				\\ \hline
	Iteração 3 & 0	\\ \hline
    \end{tabular}
    \legend{Fonte: Próprio Autor}
\end{table}

%\begin{figure}[h]
%	\centering
%	\includegraphics{img/noc.png}
%	\caption{Valores de NOC/Fonte: Próprio autor}
%	\label{fig:noc}
%\end{figure}


\subsection{DIT}

A métrica DIT está relacionada complexidade e reusabilidade. Quanto mais
abaixo na hierarquia de herança a classe estiver, menos previsível será seu
comportamento devido a quantidade de métodos que serão herdados das classes
acima nessa hierarquia. Entretanto, a métrica também indica um potencial reuso
de código por meio da herança de métodos, isso torna os parâmetros de avaliação
da métrica dependentes do contexto a ser analisado. O aumento demonstrado na DIT
se deve ao fato que o CKJM considera a implementação de interface como herança.
A interface define somente o contrato que a classe deve implementar. Sendo uma
interface, não há nenhuma implementação que possa interferir no comportamento
da classe, portanto essa alteração na métrica DIT não deve ser considerada. O
aumento no valor da métrica é mostrado na tabela~\ref{tab:dit} e
figura~\ref{fig:dit}

\begin{table}[!h]
	\centering
	    \caption{\label{tab:dit} Dados métrica DIT}
    \begin{tabular}{ | l | l | }
    \hline
    Iteração & Média 			\\ \hline
    Baseline & 0.7741935484  	\\ \hline
    Iteração 1 & 0.78125		\\ \hline
	Iteração 2 & 0.78125			\\ \hline
	Iteração 3 & 0.7878787879	\\ \hline
    \end{tabular}
    \legend{Fonte: Próprio Autor}
\end{table}

\begin{figure}[!htb]
	\caption{\label{fig:dit} Gráfico da métrica DIT}   
	\begin{center}
		\includegraphics{img/dit.png}
	\end{center}
	\legend{Fonte: Próprio Autor}
\end{figure}

\subsection{CBO}

Valores baixos de CBO são indicativos de boa modularidade e encapsulamento que
se reflete na independência da classe, o que a torna mais fácil de reutilizar,
manter e testar.
Na primeira iteração foi aplicado o padrão em um componente mais simples e foi
possível remover qualquer dependência que não fosse relacionada a interface,
diminuinido significativamente o valor da métrica. Nas iterações seguintes foram
refatoradas interfaces mais complexas onde é mais difícil desacoplar as
dependências. Algumas depedências não relacionadas a camada de apresentação
permaneceram na View, dessa forma, tanto a View como o Presenter tem referências
para essas dependências aumentando o valor da métrica. Além disso, existe o
acréscimo de duas dependências entre a View e o Presenter. A variação da métrica
é exposta na tabela \ref{tab:cbo} e na figura~\ref{fig:cbo}

\begin{table}[!h]
	\centering
	    \caption{\label{tab:cbo} Dados métrica CBO}
	
    \begin{tabular}{ | l | l | }
    \hline
    Iteração & Média 			\\ \hline
    Baseline & 10.1612903226   	\\ \hline
    Iteração 1 & 10.03125		\\ \hline
	Iteração 2 & 10.09375		\\ \hline
	Iteração 3 & 10.1515151515	\\ \hline
    \end{tabular}
    \legend{Fonte: Próprio Autor}
\end{table}

\begin{figure}[!htb]
	\caption{\label{fig:cbo} Gráfico da métrica CBO}   
	\begin{center}
		\includegraphics{img/cbo.png}
	\end{center}
	\legend{Fonte: Próprio Autor}
\end{figure}


\subsection{RFC}

Altos valores para a métrica RFC indica que uma quantidade grande de métodos são
chamados a partir de uma classe tornando-a mais complexa de testar e fazer
manutenção. Logo, esta métrica deve deve diminuir para expressar maior
qualidade no código. A métrica RFC tende a aumentar a cada iteração, o que
sugere um aumento da complexidade do código conforme é apresentado na tabela \ref{tab:rfc} e Figura
\ref{fig:rfc}.

\begin{table}[!h]
	\centering
	    \caption{\label{tab:rfc} Dados métrica RFC}
    \begin{tabular}{ | l | l | }
    \hline
    Iteração & Média 			\\ \hline
    Baseline & 23.7419354839   	\\ \hline
    Iteração 1 & 24.21875		\\ \hline
	Iteração 2 & 24.71875		\\ \hline
	Iteração 3 & 24.9090909091	\\ \hline
    \end{tabular}
    \legend{Fonte: Próprio Autor}
\end{table}

\begin{figure}[htb]
	\caption{\label{fig:rfc} Gráfico da métrica RFC}   
	\begin{center}
		\includegraphics{img/rfc.png}
	\end{center}
	\legend{Fonte: Próprio Autor}
\end{figure}


A justificativa para o aumento da métrica WMC também se aplica neste caso.
Após cada iteração de refatoração, as Views passaram a delegar
responsabilidades para o Presenter por meio de chamada de métodos, além disso, o Presenter interage com a
View da mesma forma para atualizá-la. Portanto, a quantidade de chamada de
métodos aumentaram e isso se refletiu na métrica RFC.

\subsection{LCOM}

Segundo \citeonline{cksuite} ``Um valor alto de LCOM indica uma disparidade na
funcionalidade provida pela classe.". Analisando a relação entre os métodos da
classe e seus atributos é possível dizer se a classe tem muitas
responsabilidades e é necessário dividi-la em duas ou mais classes. Esta métrica
ajuda a identificar má qualidade na estrutura do código quando os valores são
altos, apontando aumento da complexidade e pouco encapsulamento. A tabela
\ref{tab:lcom} e a figura \ref{fig:lcom} mostram uma queda siginificativa na
métrica LCOM. Isto indica que a coesão do código melhorou após cada iteração.

\begin{table}[!h]
	\centering
	    \caption{\label{tab:lcom} Dados métrica LCOM}
    \begin{tabular}{ | l | l | }
    \hline
    Iteração & Média 			\\ \hline
    Baseline & 57.4838709677   	\\ \hline
    Iteração 1 & 56.875			\\ \hline
	Iteração 2 & 53.1875		\\ \hline
	Iteração 3 & 48.4242424242	\\ \hline
    \end{tabular}
    \legend{Fonte: Próprio Autor}
\end{table}

\begin{figure}[!htb]
	\caption{\label{fig:lcom} Gráfico da métrica LCOM}   
	\begin{center}
		\includegraphics{img/lcom.png}
	\end{center}
	\legend{Fonte: Próprio Autor}
\end{figure}



\section{Discussão dos Resultados} 




%CBO
As classes destinadas à implementação da interface no framework android fornecem
acesso a recursos que servem para implementação de responsabilidades não
relacionadas com a interface. Essa característica do framework android leva à
implementação da camada de View com diversas responsabilidades que não são
inerentes à interação com o usuário. Isso dificultou a refatoração, pois as
classes que exercem o papel de Presenter necessitam interagir com as classes de
View para acessar esses recursos, além de atualizar o estado da View. O uso do
padrão de injeção de
dependência\footnote{Padrão
de projeto onde um objeto recebe as referências para as suas dependências sem
conhecer os processo de contrução das mesmas.
\url{http://en.wikipedia.org/wiki/Dependency_injection}}
pode ser aplicado para acessar esses recursos e serviços sem a necessidade de
interação com a classe de View.

Houve diminuição na métrica CBO nas classes alteradas pois diversas
responsabilidades que utilizam essas dependências foram movidas para a classe de
Presenter. Analisando de forma geral, essas dependências permanecem no pacote
além de ser criado mais um acoplamento entre a View e a nova classe Presenter.
A figura \ref{fig:classes_iteracao3} mostra a disposição das classes após a
refatoração.

\begin{figure}[htb]
	\caption{\label{fig:classes_iteracao3} Pacote após refatoração}
	\begin{center}
		\includegraphics[scale=0.5]{img/classes_iteracao3}
	\end{center}
	\legend{Fonte: Próprio Autor }
\end{figure}

%WMC/RFC OK
Ao usar o padrão MVP, a quantidade de linhas de métodos diminui pois cada um
dos componentes implementaram uma parte do caso de uso, aumentando o números de
métodos que se reflete na métrica WMC. Na implementação original onde um método
que era implementado na classe View, tinha a  responsabilidade de tratar os
eventos do usuário, acessar os componetes visuais e recuperar os dados, validar
esses dados, acessar o Model para persistir o dados e atualizar a tela como o
novo estado. Tudo isso resulta em métodos com muitas linhas de código com várias
estruturas de controle e iteração, isso aumenta a complexidade da classe.

Ao dividir as responsabilidades entre os componentes definidos no padrão, um
método complexo que era implementado na View quebrado em pelo menos três métodos
menores, para que a View possa receber as interações com o usuário e dados de
entrada para então delegar o precessamento ao Presenter que vai interagir com as
classes que exercem o papel de Model, e para finalizar, o Presenter chama algum
método da View que irá atualizar os componetes visuais com o novo estado dos
dados.

Tendo em vista que o método foi usado como unidade para calcular o WMC, essa
métrica aumentou ao aplicar o padrão MVP, pois mais métodos foram criados. A
métrica WMC mostra a complexidade de uma classe, mas apesar do aumento nos
valores, a complexidade diminuiu, ao ser usado a implementação com métodos mais
simples.
% colocar imagem de código antes e depois.
A divisão de responsabilidades também afetou negativamente a métrica RFC, pois
com o aumento da quantidade de métodos, a troca de mensagens entre componentes
e com a própria classe aumentaram.  


A métrica WMC está relacionada com as métricas DIT e NOC. Como houve pouca
variação no DIT e nenhuma variação no NOC, o aumento da WMC não tem impacto
relevante.

Os experimentos demonstraram que a aplicação do padrão MVP promoveu de forma
significativa maior coesão no aplicativo. Isso é demonstrado nos resultados da
métrica LCOM que foi a mais afetada pelo uso do padrão MVP no projeto como pode
ser observado na figura \ref{fig:allmetrics}.

\begin{figure}[htb]
	\caption{\label{fig:allmetrics} Variação das métricas ao longo das iterações}
		\includegraphics{img/allmetrics}
	\legend{Fonte: Próprio Autor}
\end{figure}


É possível observar nos resutados a relação entre LCOM e as métricas 
WMC,RFC que aumentaram conforme o LCOM diminuia. O aumento da complexidade
indicado pelas métricas WMC e RFC é pequeno em comparação ao aumento de
coesão indicada pela métrica LCOM. Pode-se chegar a essa conclusão, não somente
analisando os resultados, como também ao analisar o código em que as classes
estão menores, mais coesas, com responsabilidades bem definidas. Portanto, a
arquitetura proposta melhorou a qualidade do objeto de estudo porque promoveu
maior coesão e diminuiu a complexidade.
%


\chapter{Execução da Pesquisa}


\section{Ferramentas usadas}

O código será versionado no
Github\footnote{\url{https://github.com/diegofreitas/platform_packages_apps_contacts}}
onde será feito o gerenciamento das versões de cada iteração.
As ferramentas utilizadas para a refatoração serão a IDE Eclipse(Juno) com
plugin ADT v21 para facilitar a edição do código e ferramentas de construção
do projeto existentes no próprio repositório do android, tendo em vista que todo
o processo de compilação e empacotamento não visa ser usado em uma IDE.
Para realizar a coleta das métricas é necessário que a ferramenta analise código
java e contemple todas as métricas descritas na seção \ref{sec:metrics}. O
programa Chidamber and Kemerer Java
Metrics\footnote{\url{https://github.com/dspinellis/ckjm}} atende esses
critérios, além de ser um projeto de código-aberto.

\section{Análise do objeto de estudo}

O aplicativo a ser refatorado tem funcionalidades para gerenciamento de
contatos. Dentro deste conjunto de casos de uso será refatorado o pacote
referente ao gerenciamento de grupos de contatos presente no pacote
\textbf{com.android.contacts.group} que contém componentes de tela para
interação com o usuário, a saber:
\begin{description}
\item[GroupDetailFragment.java] Exibe os dados de um grupo de contatos.
\item[GroupBrowseListFragment.java] Fonece uma lista de grupos.
\item[GroupEditorFragment.java] Disponibiliza um formulário para edição dos
dados de um grupo.
\end{description}

Estas interfaces são usadas dentro de activities que controlam uma parte do fluxo
de interação e se comportam de forma diferente conforme o tipo de dispositivo
móvel utilizado. Devido a essa complexidade, não será feita nenhuma alteração na
interface pública dos componentes refatorados, evitando efeitos colaterais em
outras partes do aplicativo.

Os componentes elencados contêm código não somente relacionado com a lógica de
apresentação como também interagem diretamente com classes destinadas ao acesso
de dados e serviços existentes nas dependências do projeto, por exemplo,
gerenciamento de contas do usuário. 

Cada iteração consistirá na refatoração de cada um dos componentes
descritos. O marco de referência de dados das métricas presentes na tabela \ref{tab:dados_baseline} será feita a partir da
versão \verb|4.4.2_r1| do aplicativo.

\begin{table}[h]
	\centering
    \begin{tabular}{ | l | l | }
    \hline
    Métrica &	Média \\ \hline
    WMC  	&	8.5161290323   	\\ \hline
    DIT	 	&	0.7741935484	\\ \hline
	NOC  	& 	0				\\ \hline
	CBO	  	& 	10.1612903226	\\ \hline
	RFC	 	& 	23.7419354839	\\ \hline
	LCOM 	& 	57.4838709677	\\ \hline
    \end{tabular}
    \caption{Métricas CK da versão de referência}
    \label{tab:dados_baseline}
\end{table}

\section{Arquitetura Proposta}

% sempre que for referenciar as camadas (view, model, controller), utilize com primeira maiúscula. serve tb para nome de classes e padroes citados ao longo do texto

Será aplicado nos experimentos a variação do padrão MVP chamada Passive View,
pois dessa forma, o Model não precisa publicar alterações de seu estado para a
view, Logo, evita-se alterações no código referente às classes que fazem
parte da camada de Model do aplicativo de contatos. A organização do código
fonte no repositório dificulta a implementação, isto porque esses componentes estão
localizados fora do projeto afetado e são compartilhados.

As classes que extendem Fragment terão a responsabilidade da View, pois é neste
componente que a interface com o usuário é construída. A classe Activity fornece
vários métodos para recuperação de recursos de imagens, textos, inicialização de
serviços, entre outros. Isso ocorre porque a classe Activity é uma subclasse de Context, herdando diversos métodos não relacionados ao gerenciamento da interface.

Segundo \citeonline{Reenskaug:1979} ``\ldots Os papéis da View e do
Controller podem ser exercidos pelo mesmo objeto quando eles estão muito
acoplados. Exemplo: Um Menu.''(tradução livre). Porém, isso requer um boa
análise do problema em questão para decidir o nível de granularidade que esses componentes podem ter.
Portanto, é recomendável manter sempre essa separação para manter uma boa coesão
nas classes. O Presenter será uma classe auxiliar à view e pode ser implementada
como uma classe java simples. 



\section{Resultados dos Experimentos}

Esta sessão tem como objetvo mostar os resultados objetos com o processo de
refatoração aplicando o parão mvp. Cada métrica é apresentada com seus dados
para cada iteração mostrando os efeitos desses valores na qualidade.

\subsection{WMC}


A métrica WMC é usada para medir o tempo e esforço necessário para desenvolver e
manter uma classe. É recomendado que esta métrica tenha valores baixos. A
tablela \ref{tab:wmc} e a figura \ref{fig:wmc} mostram os valores dessa métricas
no projeto.

Segundo \citeonline{cksuite} “\ldots Classes com um número grande de métodos
tem uma aplicação específica limitando a possibilidade de reuso."(tradução
livre). As classes que foram refatoradas são de interface, portanto
implementam uma interação com o usuário bem específica que não serão
reutilizadas por meio de herança.
\begin{table}[h]
	\centering
    \begin{tabular}{ | l | l | }
    \hline
    Iteração & Média 			\\ \hline
    Baseline & 8.5161290323   	\\ \hline
    Iteração 1 & 8.875			\\ \hline
	Iteração 2 & 9				\\ \hline
	Iteração 3 & 8.9393939394	\\ \hline
    \end{tabular}
    \caption{Dados métrica WMC}
    \label{tab:wmc}
\end{table}

\begin{figure}[h]
	\centering
	\includegraphics{img/wmc.png}
	\caption{Gráfico da métrica WMC/Fonte: Próprio autor} 
	\label{fig:wmc}
\end{figure}


\subsection{DIT}

A métrica DIT esta relacionado complexidade e reusabilidade. Quanto mais abaixo
na hierarquia de heraça a classe estiver, menos previsível será seu
comportamento devido a quantidade de classes acima. Entretanto, a métrica também
indica um potencial reuso de código por meio da herança de métodos, isso torna
os parâmetros de avaliação da métrica dependentes do contexto a ser analisado.
O aumento demonstrado na DIT se deve ao fato que o ckjm considera a
implementação de interface como herança. A interface define somente o contrato
que a classe de implementar não havendo nenhuma implementação que pudesse
interferir no comportamento da classe, portanto essa alteração no DIT não deve
ser considerada. O aumento no valor da métrica é mostrado na
tabela~\ref{tab:dit} e Figura~\ref{fig:dit}

\begin{table}[h]
	\centering
    \begin{tabular}{ | l | l | }
    \hline
    Iteração & Média 			\\ \hline
    Baseline & 0.7741935484  	\\ \hline
    Iteração 1 & 0.78125		\\ \hline
	Iteração 2 & 0.78125			\\ \hline
	Iteração 3 & 0.7878787879	\\ \hline
    \end{tabular}
    \caption{Dados métrica DIT}
    \label{tab:dit}
\end{table}

\begin{figure}[h]
	\centering
	\includegraphics{img/dit.png}
	\caption{Valores de DIT/Fonte: Próprio autor}
	\label{fig:dit}
\end{figure}


\subsection{NOC}

Altos valores para a métrica NOC são indicativos de que existe maior reuso de
código, o esforço de testes também aumenta e maior é a probabilidade de uso
incorreto de abstração. O objetivo da métrica é variável Nenhuma classe foi
herdada para a aplicação do padrão, portanto, essa métrica permaneceu intacta
durante as iterações como pode ser observado na tabela \ref{tab:noc} e
figura~\ref{fig:noc}

\begin{table}[h]
	\centering
    \begin{tabular}{ | l | l | }
    \hline
    Iteração & Média 			\\ \hline
    Baseline & 0  	\\ \hline
    Iteração 1 & 0			\\ \hline
	Iteração 2 & 0				\\ \hline
	Iteração 3 & 0	\\ \hline
    \end{tabular}
    \caption{Dados métrica NOC}
    \label{tab:noc}
\end{table}

\begin{figure}[h]
	\centering
	\includegraphics{img/noc.png}
	\caption{Valores de NOC/Fonte: Próprio autor}
	\label{fig:noc}
\end{figure}

\subsection{CBO}

Valores baixos de CBO são indicativos de boa modularidade e encapsulamento que
se reflete na indepdendência da classe o que a torna mais fácil de reutilizar,
manter e testar.

Na primeira iteração foi aplicado o padrão em um componente mais simples e foi
possível remover qualquer dependência que não fosse relacionada a interface,
entretanto, a maior queda ocorreu em  ficando a cargo do da experiência do
desenvolvedor identificar estes cenários específicos. Nas iterações seguintes as
interfaces alteradas são mais complexos,

\begin{table}[h]
	\centering
    \begin{tabular}{ | l | l | }
    \hline
    Iteração & Média 			\\ \hline
    Baseline & 10.1612903226   	\\ \hline
    Iteração 1 & 10.03125		\\ \hline
	Iteração 2 & 10.09375		\\ \hline
	Iteração 3 & 10.1515151515	\\ \hline
    \end{tabular}
    \caption{Dados métrica CBO}
    \label{tab:cbo}
\end{table}

\begin{figure}[h]
	\centering
	\includegraphics{img/cbo.png}
	\caption{Valores de CBO/Fonte: Próprio autor}
	\label{fig:cbo}
\end{figure}


\subsection{RFC}

Altos valores para a métrica RFC indica que uma quantidade grande de métodos é
chamados a partir de uma classe tornando-a mais complexa de testar e fazer
manutenção. Logo, esta métrica deve ser mantida baixa.

A métrica RFC tende a aumentar a cada iteração o que sugere um aumento da
complexidade do código conforme é apresentado na tabela \ref{tab:rfc} e Figura
\ref{fig:rfc}.

\begin{table}[h]
	\centering
    \begin{tabular}{ | l | l | }
    \hline
    Iteração & Média 			\\ \hline
    Baseline & 23.7419354839   	\\ \hline
    Iteração 1 & 24.21875		\\ \hline
	Iteração 2 & 24.71875		\\ \hline
	Iteração 3 & 24.9090909091	\\ \hline
    \end{tabular}
    \caption{Dados métrica RFC}
    \label{tab:rfc}
\end{table}

\begin{figure}[h]
	\centering
	\includegraphics{img/rfc.png}
	\caption{Valores de RFC/Fonte: Próprio autor}
	\label{fig:rfc}
\end{figure}

\subsection{LCOM}

Esta métrica ajuda a identificar má qualidade na estrutura do código quando os
valores são altos apontando aumento da complexidade e pouco encapsulamento.
A tabela \ref{tab:lcom} e a figura \ref{fig:lcom} mostram uma queda
siginificativa na métrica LCOM.
Isto indica que a coesão do código melhorou após cada iteração. 

\begin{table}[h]
	\centering
    \begin{tabular}{ | l | l | }
    \hline
    Iteração & Média 			\\ \hline
    Baseline & 57.4838709677   	\\ \hline
    Iteração 1 & 56.875			\\ \hline
	Iteração 2 & 53.1875		\\ \hline
	Iteração 3 & 48.4242424242	\\ \hline
    \end{tabular}
    \caption{Dados métrica LCOM}
    \label{tab:lcom}
\end{table}

\begin{figure}[h]
	\centering
	\includegraphics{img/lcom.png}
	\caption{Valores de lcom/Fonte: Próprio autor}
	\label{fig:lcom}
\end{figure}


\chapter{Resultados e Conclusões}

\section{Conclusões}

É um equívoco MVC no contexto atual pois o padrão não se adequa. o que existe
são evoluções aplicadas a uma determinada situação desenvolvimento web, desktop
e 

abordar quais seriam os parâmetros aceitáveis para as métricas

Validade dessas métricas, talvez o ideal fosse customizar essas métricas para o
framework.(question mark);
Possível aumento do código com essa divisão com mais classes

quais metricas foram afetadas.

\section{Trabalhos Futuros}

Existem outros padrões de projetos para o desenvolvimento da camada de
apresentação de um software que não foram analisados nesse trabalho: MVVM,
MVP-VM, MVPC.

Este trabalho não fez uma avaliação dos impactos na performance do aplicativo
devido ao uso desses padrões. A inclusão de mais objetos interagindo,
redirecionando mensagens pode depreciar a performance.




% ---
% Finaliza a parte no bookmark do PDF, para que se inicie o bookmark na raiz
% ---
\bookmarksetup{startatroot}% 

% ----------------------------------------------------------
% ELEMENTOS PÓS-TEXTUAIS
% ----------------------------------------------------------
\postextual


% ----------------------------------------------------------
% Referências bibliográficas
% ----------------------------------------------------------
\bibliography{abntex2-modelo-references}

% ----------------------------------------------------------
% Glossário
% ----------------------------------------------------------
%
% Consulte o manual da classe abntex2 para orientações sobre o glossário.
%
%\glossary


\end{document}
